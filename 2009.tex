\chapter{2009}
\label{cha:2009}

\begin{Problema}{1}
  Tenemos 120 esferas navide�as id�nticas metidas en cajitas del mismo
  tama�o.  Cada cajita contiene el mismo n�mero de esferas y el n�mero
  de esferas en cada cajita supera en 2 al n�mero de cajitas. �Cu�ntas
  cajitas hay y cu�ntas esferas hay dentro de cada cajita?
\end{Problema}

\begin{Solucion}
  
\end{Solucion}

\begin{Problema}{2}
  Suponga que $n\geq 1$ es un n�mero natural. Suponga que $n+1$ puede
  escribirse como la suma de los cuadrados de dos n�meros
  consecutivos, es decir, $n+1=m^2+(m+1)^2$ para alg�n n�mero natural
  $m$. Demuestre que $\sqrt{2n+1}$ es un n�mero entero impar.
\end{Problema}

\begin{Solucion}
  
\end{Solucion}

\begin{Problema}{3}
  En un tri�ngulo rect�ngulo con catetos de longitud $a$ y $b$ est�
  inscrito un cuadrado de tal manera que el cuadrado y el tri�ngulo
  comparten el �ngulo recto y un v�rtice del cuadrado est� sobre la
  hipotenusa. Hallar el per�metro del cuadrado.
\end{Problema}

\begin{Solucion}
  
\end{Solucion}

\begin{Problema}{4}
  �De cu�ntas maneras diferentes se pueden dar 100 pesos en cambio
  utilizando �nicamente monedas de 10 pesos, 5 pesos y 1 peso?
\end{Problema}

\begin{Solucion}
  
\end{Solucion}

\begin{Problema}{5}
  En una fiesta cada persona salud� a exactamente otras tres personas.
  \begin{itemize}
  \item Explique por qu� es imposible que a la fiesta hayan asistido
    exactamente~2009 personas.
  \item Si hubo en total~123 saludos, �cu�ntas personas asistieron a
    la fiesta?
  \end{itemize}
\end{Problema}

\begin{Solucion}
  La primera observaci\'on es que todo saludo es rec\'iproco, es
  decir, si Juan saluda a Pedro, tambi\'en Pedro saluda a
  Juan. Entonces ya que asistieron 2009 personas, cada una salud\'o
  exactamente a tres personas, pero podemos ver que cada saludo es
  contado dos veces, retomando el ejemplo anterior, es contado una vez
  cuando Juan saluda a Pedro y una segunda vez cuando Pedro saluda a
  Juan, debido a que son el mismo caso, tenemos que dividir el
  n\'umero de saludos entre dos, por lo que deber\'ia haber
  $\tfrac{2009\times 3}{2}=3013.5$ lo cual no es posible ya que el
  n\'umero de saludos tiene que ser un n\'umero entero.

  Ahora para el segundo inciso, realizaremos un razonamiento similar,
  si el total de saludos es 123, al multiplicar por dos \'este
  n\'umero deber\'ia ser igual al n\'umero de personas multiplicado
  por tres, retomando lo que se aplic\'o para resolver el inciso
  anterior, por lo tanto $123\times 2=246$ debe ser el triple de las
  personas que asistieron a la fiesta, por lo que hay
  $\tfrac{246}{3}=82$ personas, lo que se puede comprobar realizando
  el mismo proceso que en el inciso anterior para encontrar el
  n\'umero de saludos conociendo el n\'umero de personas.
\end{Solucion}

\begin{Problema}{6}
  En una granja hay un granero cuadrangular de 20 metros de
  lado. Una vaca est� atada a una de las paredes del granero, a 5
  metros de una esquina, con una soga de 10 metros de longitud.
  Calcule la superficie total sobre la cual la vaca puede pastar (la
  vaca s�lo puede pastar fuera del granero).
\end{Problema}

\begin{Solucion}
  
\end{Solucion}

%%% Local Variables: 
%%% mode: latex
%%% TeX-master: "libro"
%%% End: 