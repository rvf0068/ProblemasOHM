\chapter{2013}
\label{cha:2013}

\section{Primera parte}
\label{sec:primera-parte2013}

\begin{Problema}{1}
  ?`Cu\'al es el \'ultimo d\'igito del n\'umero $(2013)^{2014}$?

  \begin{inparaenum}
  \item 1 \espl
  \item 3 \espl
  \item 5 \espl
  \item 7 \espl
  \item 9
  \end{inparaenum}
\end{Problema}

\begin{Problema}{2}
  Un c\'irculo de radio $1$ est\'a inscrito en un tri\'angulo
  equil\'atero. El tri\'angulo, a su vez, est\'a ins\-cri\-to en otro
  c\'irculo.  Calcula el \'area entre los dos c\'irculos (el \'area
  sombreada en la figura).

  \begin{inparaenum}
  \item $\pi$ \espm
  \item $\sqrt{3} \pi$ \espm
  \item $2 \pi$ \espm
  \item $3 \pi$ \espm
  \item \nota.
  \end{inparaenum}

%  \begin{figure}[!htbp]
  % \begin{center}
  %   \begin{pspicture}(-1.6,-1.6)(1.6,1.6)\psset{unit=0.8cm}
  %     \pscircle[fillstyle=solid,fillcolor=lightgray]{2}\pscircle[fillstyle=solid,fillcolor=white]{1}
  %     \pspolygon(0,2)(1.732,-1)(-1.732,-1)(0,2)\psline[linestyle=dashed]{*->}(1,0)\uput[d](0.5,0){$1$}
  %   \end{pspicture}
  %   \end{center}
%  \end{figure}
\end{Problema}

\begin{Problema}{3}
Se tiene una cuarta parte de un c\'irculo de radio $2$. Inscrito
  en \'el, como se ve en la figura, est\'a otro c\'irculo. ?`Cual es el per\'imetro del c\'irculo peque\~no?

\begin{inparaenum}
\item $\pi$ \espc
\item $2 \pi$ \espc
\item $\frac{2}{1+\sqrt{2}} \pi$ \espm
\item $\frac{4}{1+\sqrt{2}} \pi$ \espm
\item \nota.
\end{inparaenum}

% \begin{figure}[!htbp]
% \centering\begin{pspicture}(0,0)(1.6,1.6)\psset{unit=0.8cm}
%   \pswedge[fillstyle=solid,fillcolor=lightgray]{2}{0}{90}\pscircle[fillstyle=solid,fillcolor=white](0.828427,0.828427){0.828427}
%   \psline(-0.4,2)(-0.2,2)\psline(-0.4,0)(-0.2,0)\psline[linestyle=dashed]{<->}(-0.3,2)(-0.3,0)\uput[l](-0.3,1){$2$}
% \end{pspicture}
% \end{figure}
\end{Problema}

\begin{Problema}{4}
  En Actopan hay una cerca con forma de tri\'angulo equil\'atero
  que mide $6$ metros por lado y que protege un campo de
  alfalfa. Afuera del tri\'angulo hay un borrego atado 
 con una cuerda de $3$ metros de longitud a un poste de la cerca. Si el poste
  est\'a a $2$ metros de una de las esquinas, ?`cu\'al es el tama\~no
  del \'area total en la que el borrego puede pastar?


\begin{inparaenum}
\item $5\pi$ \espm
\item $\frac{5}{3}\pi$ \espm
\item $\frac{29}{5} \pi$ \espm
\item $\frac{29}{6} \pi$ \espm
\item \nota.
\end{inparaenum}
\end{Problema}

\begin{Problema}{5}
  Calcula el valor de
$$
\frac{123456789}{(-123456789)^2+(-123456788)(123456790)}
$$

\begin{inparaenum}
\item $123456788$ \espsc
\item $123456789$ \espsc
\item $123456790$ \espsc
\item $\frac{1}{2\times123456789}$ \espsc
\item \nota.
\end{inparaenum}
\end{Problema}

\begin{Problema}{6}
  Dadas $6$ rectas distintas en el plano, ?`cu\'al es el n\'umero
  m\'aximo de puntos en los que pueden intersectarse?

\begin{inparaenum}
\item $6$ \espm
\item $10$ \espm
\item $15$ \espm
\item $16$ \espm
\item \nota.
\end{inparaenum}
\end{Problema}

\begin{Problema}{7}
Se colocan $7$ c\'irculos de radio $1$ de manera que los centros
  est\'en en linea recta y sean tangentes, como se muestra en la
  figura. Calcula el \'area de la regi\'on sombreada.


\begin{inparaenum}
\item $24+\pi$ \espc
\item $24-6\pi$ \espc
\item $24-7\pi$ \espc
\item $28 -7\pi$ \espc
\item \nota.
\end{inparaenum}

% \begin{figure}[!htbp]
% \centering\begin{pspicture}(-6,-0.8)(6,0.8)\psset{unit=0.8cm}
%   \psframe[fillstyle=solid,fillcolor=lightgray](-6,-1)(6,1)
%   \pscircle[fillstyle=solid,fillcolor=white](0,0){1}\pscircle[fillstyle=solid,fillcolor=white](2,0){1}
%   \pscircle[fillstyle=solid,fillcolor=white](4,0){1}\pscircle[fillstyle=solid,fillcolor=white](6,0){1}
%   \pscircle[fillstyle=solid,fillcolor=white](-2,0){1}\pscircle[fillstyle=solid,fillcolor=white](-4,0){1}
%   \pscircle[fillstyle=solid,fillcolor=white](-6,0){1}\psline(-7.5,0)(7.5,0)\psline[linestyle=dashed]{*->}(0,0)(0,1)\uput[l](0,0.5){$1$}
%   \psline[linestyle=dashed]{*->}(2,0)(2,1)\uput[l](2,0.5){$1$}\psline[linestyle=dashed]{*->}(4,0)(4,1)\uput[l](4,0.5){$1$}
%   \psline[linestyle=dashed]{*->}(6,0)(6,1)\uput[l](6,0.5){$1$}\psline[linestyle=dashed]{*->}(-2,0)(-2,1)\uput[l](-2,0.5){$1$}
%   \psline[linestyle=dashed]{*->}(-4,0)(-4,1)\uput[l](-4,0.5){$1$}\psline[linestyle=dashed]{*->}(-6,0)(-6,1)\uput[l](-6,0.5){$1$}
% \end{pspicture}
% \end{figure}
  
\end{Problema}

\begin{Problema}{8}
  Considera un n\'umero de tres d\'igitos $abc$ y un n\'umero de
  dos d\'igitos $aa$. Si multiplicas estos dos n\'umeros se obtiene el $2013$. ?`Cu\'al
  es el valor de $a \times b \times c$?

\begin{inparaenum}
\item $30$ \espm
\item $183$ \espm
\item $671$ \espm
\item $2013$ \espm
\item \nota.
\end{inparaenum}
\end{Problema}

\begin{Problema}{9}
  En casa hay tres relojes. El 9 de febrero de 2013 a las 10:00am
  todos ellos indicaban la hora correctamente, pero solo marchaba bien
  el primer reloj. El segundo se atrasaba un minuto al d\'ia y el
  tercero se adelantaba un minuto al d\'ia. Si los relojes contin\'uan
  marchando as\'i, ?`al cabo de cu\'anto tiempo volver\'an los tres a
  marcar exactamente las 10:00am?

\begin{inparaenum}
\item $240$ d\'ias \espsc
\item $720$ d\'ias \espsc
\item $1440$ d\'ias \espsc
\item nunca lo har\'an \espsc
\item \nota.
\end{inparaenum}
\end{Problema}


\begin{Problema}{10}
La base de un rect\'angulo es el doble de su altura. Si la base
  se disminuye en $6$ unidades y la altura se aumenta en $4$ el \'area
  del rect\'angulo no cambia. ?`Cu\'al es el \'area del rect\'angulo?

\begin{inparaenum}
\item $1$ \espm
\item $24$ \espm
\item $144$ \espm
\item $288$ \espm
\item \nota.
\end{inparaenum}
\end{Problema}

\section{Segunda parte}
\label{sec:segunda-parte2013}

\begin{Problema}{11}
Decimos que tres enteros positivos $a$, $b$ y $c$ est\'an en
  progresi\'on aritm\'etica si $a < b < c$ y $b-a=c-b$. Supongamos que
  $a$, $b$ y $c$ son enteros positivos en progresi\'on
  aritm\'etica. Demuestra que
$$
\frac{1}{\sqrt{b}+\sqrt{c}}, \quad \frac{1}{\sqrt{c}+\sqrt{a}}, \quad \frac{1}{\sqrt{a}+\sqrt{b}}
$$
est\'an en progresi\'on aritm\'etica.
\end{Problema}

\begin{Problema}{12}
  Calcula el valor de
$$
(2+1)(2^2+1)(2^4+1)(2^8+1)(2^{16}+1)(2^{32}+1)(2^{64}+1)(2^{128}+1)(2^{256}+1)(2^{512}+1)(2^{1024}+1) + 1.
$$
\end{Problema}

\begin{Problema}{13}
En una cuarto de tama\~no rectangular se acomodan $m \times n$ sillas de
  manera rectangular de tal modo que se forman $m>1$ filas y $n>1$
  columnas de sillas. Entran los estudiantes al examen y cada uno se
  sienta en una silla, sin que queden sillas vac\'ias. Despu\'es, cada
  estudiante saluda de mano a los que est\'an junto a \'el (al que
  est\'a a su derecha, a su izquierda, adelante y atr\'as). Si en total
  se hicieron $275$ saludos, ?`cu\'antas sillas hay en el sal\'on?
\end{Problema}

%%% Local Variables: 
%%% mode: latex
%%% TeX-master: "libro"
%%% End: 
