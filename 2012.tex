\chapter{2012}
\label{cha:2012}

\section{Primera parte}
\label{sec:primera-parte2012}

\begin{Problema}{1}
  El precio de un pastel era de $200$ pesos, pero subi\'o un
  $25\%$. Sobre ese nuevo precio, se puso en oferta. Result\'o que con
  la oferta, el pastel volvi\'o a costar $200$ pesos. ?`De cu\'anto
  fue la oferta?

  \begin{inparaenum}
  \item $23\%$ \esp
  \item $25\%$ \esp
  \item $20\%$ \esp
  \item $35\%$ \esp
  \item De otro porcentaje
  \end{inparaenum}
\end{Problema}

\begin{Solucion}
  El pastel primero subi� $(0.25)(200)=\frac{200}{4}=50$ pesos, por lo
  que el nuevo precio era $200+50=250$ pesos. Queremos ahora saber qu�
  porcentaje de 250 es 50 pesos. Puesto que el $x\%$ de 250 es
  $250(\frac{x}{100})$, debemos resolver $250(\frac{x}{100})=50$, de
  donde se obtiene $x=20$. Es decir, la oferta fue de un $20\%$.
\end{Solucion}

\begin{Problema}{2}
  En un torneo de ping-pong hay ocho jugadores. Si cada uno de ellos
  juega una vez contra cada uno de los otros siete, ?`cu\'antos
  partidos se juegan en total?

  \begin{inparaenum}
  \item $8$ \esp
  \item $56$ \esp
  \item $28$ \esp
  \item $49$ \esp 
  \item Otro n\'umero
  \end{inparaenum}
\end{Problema}

\begin{Solucion}
  Como cada jugador juega contra otros siete, en principio ser�an
  $8\cdot 7=56$ partidos. Pero el partido del jugador A contra el
  jugador B es el mismo que el del jugador B contra el jugador A, por
  lo que en realidad la cantidad de partidos es $\frac{8\cdot
    7}{2}=28$. 
\end{Solucion}

\begin{Problema}{3}
  Considera la sucesi\'on de numeros $1, 3, 7, 13, 21, \cdots\,$ (los
  puntos suspensivos significan que es una lista infinita de
  n\'umeros, por lo que no la podemos escribir toda). El n\'umero que
  aparece en el lugar $n$ lo puedes calcular con la f\'ormula
  $n^2-n+1$ (por ejemplo, el n\'umero que aparece en el quinto lugar
  es $5^2-5+1=21$). ?`Qu\'e n\'umero le sigue al $91$ en la
  sucesi\'on?

  \begin{inparaenum}
  \item El $91$ no est\'a en la sucesi\'on \esp
  \item $98$ \esp
  \item $109$ \esp
  \item $111$ \esp 
  \item Otro n\'umero distinto de los anteriores 
\end{inparaenum}
\end{Problema}

\begin{Solucion}
  Las solucion a la ecuaci�n cuadr�tica $n^{2}-n+1=91$ son $n=10$,
  $n=-9$, por lo que 91 ocupa el �ndice 10 en la sucesi�n, y el
  siguiente n�mero corresponde al $n=11$, es decir $11^{2}-11+1=111$.
\end{Solucion}

\begin{Problema}{4}
  Tres tristes tigres tragaban trigo en treinta tristes trastos en un
  trigal.  El primer tigre trag\'o el doble de tristes trastos de
  trigo que el segundo tigre, mientras que el segundo tigre trag\'o el
  triple de tristes trastos de trigo que el tercer tigre.  ?`Cu\'antos
  tristes trastos de trigo trag\'o el tercer tigre?

  \begin{inparaenum}
  \item $15$ \esp
  \item $10$ \esp
  \item $5$ \esp
  \item $20$ \esp
  \item Otro n\'umero de trastos
  \end{inparaenum}
\end{Problema}

\begin{Solucion}
  Si el tercer tigre trag� $x$ trastos, el segundo trag� $3x$ y el
  primero trag� $2(3x)=6x$. Es decir, en total se tragaron
  $6x+3x+x=10x=30$ trastos, por lo que $x=3$.
\end{Solucion}

\begin{Problema}{5}
  Considera un rect\'agulo cuyas longitudes de sus lados son n\'umeros enteros. Si 
  el \'area del rect\'angulo es 21, ?`cu\'ales son los posibles valores de sus lados?

  \begin{inparaenum}
  \item $20$ y $44$ \quad\qquad
  \item $15$ y $30$ \quad\qquad
  \item $17$ y $44$ \quad\qquad
  \item $10$ y $22$ \quad\qquad
  \item Otro par de valores
  \end{inparaenum}
\end{Problema}

\begin{Solucion}
  Los �nicos posibles valores para las longitudes de los lados son:
  21,1 o 3,7.
\end{Solucion}

\begin{Problema}{6}
  En un c\'irculo se encuentra inscrito un tri\'angulo equil\'atero, como en la figura. 
  Si el \'area del tri\'angulo es uno, ?`cu\'al es el valor del \'area de la regi\'on 
  sombreada?

  \begin{center}
    \includegraphics[scale=0.20]{critri.pdf}
  \end{center}

  \begin{inparaenum}
  \item $\dfrac{4\pi}{3\sqrt{3}}-1$, \quad\qquad
  \item $\dfrac{4\pi}{\sqrt{3\sqrt{3}}}-1$ \quad\qquad
  \item $\dfrac{4\pi}{9}-1$ \quad\qquad
  \item $\dfrac{4\pi}{3}-1$ \quad\qquad 
  \item Otro valor
  \end{inparaenum}
\end{Problema}

\begin{Solucion}
  Sea $r$ el radio del c�rculo. Queremos entonces expresar el �rea del
  tri�ngulo en funci�n de $r$, y para eso necesitamos a su vez
  expresar la base y la altura en funci�n de $r$.

  Partiendo del centro del c�rculo y trazando una perpedicular a un
  lado, formamos un peque�o tri�ngulo rect�ngulo que tiene un radio
  $r$ como hipotenusa. Dado que el tri�ngulo original es equil�tero,
  el segmento perpendicular trazado mide $\frac{r}{2}$, y por lo tanto
  el otro cateto mide $\frac{r\sqrt{3}}{2}$. Dado que este �ltimo
  cateto mide la mitad del lado del tri�ngulo original, tenemos que el
  tri�ngulo original tiene lado $r\sqrt{3}$ y altura $\frac{3r}{2}$,
  de donde se obtiene:
  \begin{equation}
    \label{eq:5}
    r^{2}=\frac{4}{3\sqrt{3}}.
  \end{equation}

  Por lo tanto, el �rea del c�rculo es $\pi\frac{4}{3\sqrt{3}}$, y el
  �rea sombreada es $\frac{4\pi}{3\sqrt{3}}-1$.
\end{Solucion}

\begin{Problema}{7}
  El ayuntamiento de Mineral de la Reforma te ha encargado pintar las
  cuatro caras exteriores de un monumento piramidal como el que se
  muestra en la figura. La pir\'amide es lisa, tiene una base cuadrada
  de 16 metros de lado y una altura de 6 metros. Una cubeta de pintura
  cuesta 900 pesos y con ella puedes pintar 40 metros
  cuadrados. ?`Cu\'anto cuesta toda la pintura que necesitas?

  \begin{center}
    \includegraphics[scale=.44]{OHMpyramid.png}
  \end{center}

  \begin{inparaenum}
  \item $\$7,200$ \quad\qquad
  \item $\$12,960$ \quad\qquad
  \item $\$4,320$ \quad\qquad
  \item $\$10,800$ \quad\qquad
  \item Otra cantidad
  \end{inparaenum}
\end{Problema}

\begin{Solucion}
  Trazando una l�nea perpendicular desde el centro de la base de la
  pir�mide al lado de la base marcado con 16, se obtiene un tri�ngulo
  rect�ngulo con catetos que miden 6 y 8, por lo que la hipotenusa
  mide $\sqrt{6^{2}+8^{2}}=10$. Se deduce que la superficie a pintar
  est� compuesta por cuatro tri�ngulos de base 16 y altura 10, por lo
  que el total a pintar es $4\frac{16\cdot 10}{2}=320$ metros
  cuadrados. Por lo tanto, se necesitan $\frac{320}{40}=8$ cubetas de
  pintura, que cuestan en total $8\cdot 900=7,200$ pesos.
\end{Solucion}

\begin{Problema}{8}
  ?`Cu\'al es el valor m\'as grande que puede tomar la cantidad 
  $\displaystyle{(m+n)/n}$, si sabes que $m$ y $n$ son n\'umeros enteros positivos 
  tales que $n>99$ y $m<101$?

  \begin{inparaenum}
  \item 2 \esp
  \item 7 \esp
  \item 38 \esp
  \item 5 \esp
  \item Ninguno de los valores anteriores
  \end{inparaenum}
\end{Problema}

\begin{Solucion}
  Tenemos que $\frac{m+n}{n}=\frac{m}{n}+1$. Una fracci�n
  vale m�s cuando su numerador grande y su denominador es
  peque�o. Como $m$ vale a lo m�s 100 y $n$ vale al menos 100, tenemos
  que $\frac{m}{n}+1$ vale a lo m�s $\frac{100}{100}+1=2$.
\end{Solucion}

\begin{Problema}{9}
  En una tina con agua como la de la figura, flota una pelota de $60$
  cent\'imetros de di\'ametro. Si el di\'ametro de la tina es de $180$
  cent\'imetros y exactamente un cuarto del volumen de la pelota se
  encuentra sumergida en la tina, ?`cu\'antos cent\'imetros bajar\'a
  el nivel del agua en el recipiente cuando se saque la pelota? El
  volumen total de la pelota es de $36,000\pi$ cm$^3$.

  \begin{center}
    \includegraphics[scale=.58]{OHMballinpool.png}
  \end{center}

  \begin{inparaenum}
  \item $1$ cm\esp
  \item $\dfrac{\pi}{3}$ cm \esp
  \item $\dfrac{10}{9} $ cm\esp
  \item $\dfrac{\pi}{2}$\esp
  \item Otra cantidad
  \end{inparaenum}  
\end{Problema}

\begin{Solucion}
  
\end{Solucion}

\begin{Problema}{10}
  Chaco planea robar un banco y le promete darle al Roto un tercio del
  bot\'in si le ayuda. El Roto acepta, pero como no sabe manejar, le
  pide ayuda al Pepis a cambio de darle un tercio de lo que ganar\'a
  por el robo. El Pepis acepta, pero su auto est\'a en el taller,
  as\'i que le dice al Chueco que si le deja usar su auto, le dar\'a
  un tercio de lo que \'el gane. El Chueco acepta, pero como no oye
  bien, le dice al Pollo que si le ayuda le dar\'a un tercio de lo que
  gane. El Pollo acepta. Tres d\'ias despu\'es del robo atrapan al
  Chueco con su parte del bot\'in, que es de 40,000 pesos.  ?`Cu\'anto
  dinero se llevaron entre todos?

  \begin{inparaenum}
  \item \$810,000 \quad\quad
  \item \$3,240,000 \quad\quad
  \item \$1,620,000 \quad\quad
  \item \$1,000,000 \quad\quad 
  \item Otra cantidad
  \end{inparaenum}
\end{Problema}

\begin{Solucion}
  
\end{Solucion}

\section{Segunda parte}
\label{sec:segunda-parte2012}

\begin{Problema}{11}
  Paco y Luis se tienen que formar en una fila con sus compa\~neros
  Carlos, Miguel, Daniel y Joel. ?`De cu\'antas formas distintas se
  pueden acomodar de manera que entre Luis y Paco se encuentren
  formados exactamente dos de sus compa\~neros?
\end{Problema}

\begin{Solucion}
  
\end{Solucion}

\begin{Problema}{12}
  Si $c$ es la longitud de la hipotenusa de un tri\'angulo
  rect\'angulo cuyos otros dos lados tienen longitudes $a$ y $b$,
  muestra que $a+b\leq\sqrt{2}c$.
\end{Problema}

\begin{Solucion}
  
\end{Solucion}

\begin{Problema}{13}
  Considera un c\'irculo con centro $O$, como se muestra en la figura.
  $DEB$ es una cuerda del c\'irculo tal que $DE=3$ y $EB=5$. Si $AC$
  es un di\'ametro del c\'irculo que pasa por el punto $E$ y $EC=1$,
  ?`cu\'anto debe medir el radio del c\'irculo?

  \begin{center}
    \includegraphics[scale=0.85]{problema2.png}
  \end{center}
\end{Problema}

\begin{Solucion}
  
\end{Solucion}


%%% Local Variables: 
%%% mode: latex
%%% TeX-master: "libro"
%%% End: 